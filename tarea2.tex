\documentclass[letter]{article}

\usepackage{MD_estilo}

\nombre{Felipe Haase Vargas} % Aqui va el nombre del alumno
\numtarea{2} % Aqui va el número de la tarea


\begin{document}
    
    \begin{pregunta}{1} % Aqui se coloca el número de la pregunta
            $$ I_{w}(dom) := N $$
            $$ I_{w}(\leq) := \text{ orden sobre los naturales} $$
            $$ I_{w}(A) := A_{i} = 1 \Leftrightarrow x_{i} = a $$
            $$ I_{w}(B) := B_{i} = 1 \Leftrightarrow x_{i} = b $$
            $$ I_{w}(C) := C_{i} = 1 \Leftrightarrow x_{i} = c $$
            1)\\
            $$ \phi := \forall x \exists y (x \leq y \wedge A(y))$$
            Se cumple ya que los naturales son infinitos, por lo tanto si para todos los x,\\
            que son infinitos, se cumple A(x), entonces todas las posiciones de la palabra\\
            son "a"\\\\
            2)\\
            $$ 0 := \exists y \forall x (y \leq x ) \text{ Con esto definimos el 0}$$
            $$ L(x,y) := \leq y \wedge (y \leq x) \wedge (x \leq y) $$
            $$ S(x,y) := L(x,y) \wedge \neg \exists z. (L(x,z) \wedge L(z,y)) $$
            $$ \exists y \exists z. (L(0,y) \wedge L(y,z) \wedge A(0) \wedge B(y) \wedge C(z) $$
            Funciona porque definimos el sucesor de los numeros, y tambien al 0, luego decimos\\
            que en 0 es a, en el sucesor de 0 es b y en el sucesor del sucesor de 0 es 1.
            \\\\
            3)\\\\
            $$ \phi := \exists x \exists y \exists z. (\neg (y \leq x) \wedge \neg (z \leq y) \rightarrow A(x) \wedge B(y) \wedge C(z) )  $$
            4)\\\\
            $$ \forall x. (( x \leq 0 \rightarrow A(x) \wedge B(y) )\lee ) $$
    \end{pregunta}
    
    \begin{pregunta}{2}
        2.1)\\\\
        $$(\leftarrow)$$
        $$ (\exists x. \phi(x)) \wedge (\exists x. \psi(x)) \equiv \exists y.\exists z.(\phi(y) ∧ \psi(z))$$
        $$ \text{Esto quiere decir que existen valores a,b en el dominio que cumplen lo enunciado} $$
        $$ $$
        2.2)\\\\
        $$ \text{Demostrando por induccion se tiene que } $$
        $$ \text{Para k = 0} $$
        $$ \phi_{0} := \psi \equiv \phi^{'}_{0} := \psi $$
        $$ \text{para k = n, suponemos que es verdad: } $$
        $$ \phi_{n} := H(x_{1},..,x_{n})\psi \equiv \phi_{n}^{'} := H^{'}(x)\psi $$
        $$ \text{Con } H(x_{1},..,x_{n}) = Q_{1}x_{1}...Q_{n}x_{n} \text{ y } H^{'}(x) \text{ una variacion de cuantificadores valida}$$
        $$ \text{Luego para k = n+1}$$
        $$ \phi_{n+1} := H(x_{1},..,x_{n})Q_{n+1}x_{n+1} \equiv \phi_{n+1}' := H'(x_{1},..,x_{n})Q_{n+1}x_{n+1}\psi $$
        \qed
    \end{pregunta}

\end{document}